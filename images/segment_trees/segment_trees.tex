%%%%%%%%%%%%%%%%%%%%%%%%%%%%%%%%%%%%%%%%%%%%%%%%%%%%%%%%%%%%%%%%%%%%%%%%%%%%%
% Plantilla de Latex para trabajos de informática.
%
% Autor: Andrés Herrera Poyatos (https://github.com/andreshp) 
%         (https://www.researchgate.net/profile/Andres_Poyatos)
%
% Esta plantilla utiliza el paquete Minted para resaltar código.
% Este paquete necesita una instalación previa. Para más información
% ver:http://www.ctan.org/tex-archive/macros/latex/contrib/minted
%
% La plantilla se encuentra adaptada al español.
%
%%%%%%%%%%%%%%%%%%%%%%%%%%%%%%%%%%%%%%%%%%%%%%%%%%%%%%%%%%%%%%%%%%%%%%%%%%%%%

%----------------------------------------------------------------------------------------
%	PAQUETES Y OTRAS CONFIGURACIONES DEL DOCUMENTO
%----------------------------------------------------------------------------------------

\RequirePackage[l2tabu, orthodox]{nag}  % Produce un warnig en caso de usar un comando obsoleto.

\documentclass{article}

% Paquetes para el diseño de página:
\usepackage{fancyhdr}                   % Utilizado para hacer títulos propios.
\usepackage{lastpage}                   % Referencia a la última página. Utilizado para el pie de página.
\usepackage{extramarks}                 % Marcas extras. Utilizado en pie de página y cabecera.
\usepackage[parfill]{parskip}           % Crea una nueva línea entre párrafos.
\usepackage[margin=3cm]{geometry}       % Asigna la "geometría" de las páginas.
\usepackage{lipsum}                     % Texto dummy. Quitar en el documento final.

% Fuente utilizada. Elija uno de ellos:
\usepackage{courier}                    % Fuente Courier.
%\usepackage{fourier}                   % Fuente Adobe Utopia.
\usepackage{microtype}                  % Mejora la letra final de cara al lector.

% Paquetes para imágenes:
\usepackage[usenames,dvipsnames]{color} % Permite crear colores propios. Utilizado para el bg de Minted.
\usepackage{graphicx}                   % Utilizado para insertar gráficos.
\usepackage{caption}                    % Títulos y leyendas para los gráficos.
\usepackage{subcaption}                 % Subfiguras.
\usepackage{layout}                     % Introduce una imagen con el diseño de la página. Quitar en el documento.
\usepackage{tikz}
\usetikzlibrary{calc}

% Paquetes para la insercción y el resaltado del código:
\usepackage{minted}                     % Insercción y resaltado de código con Minted.

% Paquetes para matemáticas:
\usepackage{amsthm}                     % Teoremas y definiciones.

% Paquetes para tablas:
\usepackage{booktabs}

% Paquetes para adaptar Látex al Español:
\usepackage[spanish,es-noquoting, es-tabla, es-lcroman]{babel} % Cambia 
\usepackage[utf8]{inputenc}                                    % Permite los acentos.
\selectlanguage{spanish}                                       % Selecciono como lenguaje el Español.

% Estilo de página:
\pagestyle{fancy}                      % fancy
\fancyhf{}                             % Limpia la cabecera y el pie de página.
\geometry{left=3cm,right=3cm,top=3cm,bottom=3cm,headheight=1cm,headsep=0.5cm} % Márgenes y cabecera.

% Espacios en el documento:
\linespread{1.1}                        % Espacio entre líneas.
\setlength\parindent{0pt}               % Selecciona la indentación para cada inicio de párrafo.

% Cabecera del documento:
\renewcommand\headrule{                 % Se ajusta la línea de la cabecera.
\begin{minipage}{1\textwidth}           % Elija una de las siguientes líneas:
%    \hrule width \hsize \kern 1mm \hrule width \hsize height 2pt 
    \hrule width \hsize 
\end{minipage}
}
\lhead{\autor}                          % Parte izquierda.
\chead{}                                % Centro.
\rhead{\titulo}                         % Parte derecha.

% Pie de página del documento:
\renewcommand\footrule{                                 % Se ajusta la línea del pie de página.
\begin{minipage}{1\textwidth}                           % Elija una de las siguientes líneas:
%    \hrule width \hsize height 2pt \kern 1mm \hrule width \hsize   
    \hrule width \hsize   
\end{minipage}\par
}
\lfoot{}                                                 % Parte izquierda.
\cfoot{}                                                 % Centro.
\rfoot{Página\ \thepage\ de\ \protect\pageref{LastPage}} % Parte derecha.

%----------------------------------------------------------------------------------------
%	CONFIGURACIÓN DE LA INCLUSIÓN DE CÓDIGO
%----------------------------------------------------------------------------------------

\usemintedstyle{autumn}                      % Se elige el estilo para minted.
\definecolor{bg}{rgb}{0.95,0.95,0.95}        % Se define el color bg usado para bgcolor de Minted.
\renewcommand\listingscaption{Código}        % Se redefine el nombre dado a un bloque de código.
\renewcommand\listoflistingscaption{Códigos} % Se redefine el nombre dado a la lista de códigos.

%----------------------------------------------------------------------------------------
%	ESTRUCTURA DEL DOCUMENTO
%----------------------------------------------------------------------------------------

\newtheorem{definition}{Definición}[section]  % Se crea la subsección "Definición". 
\setcounter{secnumdepth}{2}                   % Se enumeran las secciones con profundidad 2.

%----------------------------------------------------------------------------------------
%	NUEVOS COMANDOS
%----------------------------------------------------------------------------------------

% Portada:
\newcommand{\titulo}{Título}  % Título del trabajo.
\newcommand{\fecha}{1 \ de \ Enero \ de \ 2015}                         % Fecha.
\newcommand{\asignatura}{Latex}                                         % Asignatura.
\newcommand{\autor}{Andrés Herrera Poyatos}                             % Autor.

%----------------------------------------------------------------------------------------
%	PORTADA 
%----------------------------------------------------------------------------------------

\title{                                             % Título
    \vspace{2in}
    \textmd{\textbf{\asignatura \\ \titulo}} \\         % - Nombre del trabajo
    \normalsize\vspace{0.1in}\small{\fecha}  \\         % - Fecha (Arriba) 
    \vspace{3in}
}

\author{\textbf{\autor}}                            % Autor
\date{}                                             % Fecha. Elija entre esta y la del título.

%----------------------------------------------------------------------------------------

\begin{document}

\maketitle

%----------------------------------------------------------------------------------------
%	ÍNDICE
%----------------------------------------------------------------------------------------

% Profundidad del Índice:
%\setcounter{tocdepth}{1}

\newpage
\tableofcontents
\listoflistings      % Lista con los códigos. Borrar si se desea.
\newpage

%----------------------------------------------------------------------------------------
%	Sección 1: Listing y Minted
%----------------------------------------------------------------------------------------

\begin{section}{Primera Sección}

\begin{subsection}{Árboles}

\begin{figure}[H]
\caption{Segment tree generado sobre el vector [3,2,8,5,6,1,7,4]}
\centering
\begin{tikzpicture}[auto, 
       every node/.style={fill=blue!10, circle, minimum size=15mm, draw=blue, thick, 
            text centered,inner sep=0pt}, 
       level 1/.style={sibling distance=90mm, level distance=2.5cm},
       level 2/.style={sibling distance=40mm, level distance=2.5cm}]
       level 3/.style={sibling distance=2mm, level distance=2.5cm}]
\node[draw](z){V[1,8]}
    child{ 
        node {V[1,4]}
        child{
            node {V[1,2]} 
            child{
                node [right = 3mm] {V[1,1]} }
            child{
                node [left = 3mm] {V[2,2]} }            
        }
        child{
            node {V[3,4]} 
            child{
                node [right = 3mm] {V[3,3]} }
            child{
                node [left = 3mm] {V[4,4]} }        
        }
    } 
    child{ 
        node {V[5,8]}
        child{
            node {V[5,6]} 
            child{ node [right = 3mm] {V[5,5]} }
            child{ node [left = 3mm] {V[6,6]} }
        }
        child{
            node {V[7,8]} 
            child{
                node [right = 3mm] {V[7,7]} }
            child{
                node [left = 3mm] {V[8,8]} }
        }
    };
\end{tikzpicture}
\end{figure}
\label{fig:malware}

\begin{figure}[H]
\caption{Segment tree generado sobre el vector [3,2,8,5,6,1,7,4] representado como heap}
\centering
\begin{tikzpicture}
    \coordinate (s) at (0,0);
    \node[minimum size=6mm, minimum width = 15mm, draw, rectangle] at (s) {V[1,8]};
    \coordinate (s) at ($(s) + (0,0.5)$);
    \node[minimum size=4.5mm, minimum width = 15mm, draw, rectangle] at (s) {1};
    \coordinate (s) at ($(s) + (1.5,-0.5)$);
    \node[minimum size=6mm, minimum width = 15mm, draw, rectangle] at (s) {V[1,4]};
    \coordinate (s) at ($(s) + (0,0.5)$);
    \node[minimum size=4.5mm, minimum width = 15mm, draw, rectangle] at (s) {2};
    \coordinate (s) at ($(s) + (1.5,-0.5)$);
    \node[minimum size=6mm, minimum width = 15mm, draw, rectangle] at (s) {V[4,8]};
    \coordinate (s) at ($(s) + (0,0.5)$);
    \node[minimum size=4.5mm, minimum width = 15mm, draw, rectangle] at (s) {3};
    \coordinate (s) at ($(s) + (1.5,-0.5)$);
    \node[minimum size=6mm, minimum width = 15mm, draw, rectangle] at (s) {V[1,2]};
    \coordinate (s) at ($(s) + (0,0.5)$);
    \node[minimum size=4.5mm, minimum width = 15mm, draw, rectangle] at (s) {4};
    \coordinate (s) at ($(s) + (1.5,-0.5)$);
    \node[minimum size=6mm, minimum width = 15mm, draw, rectangle] at (s) {V[3,4]};
    \coordinate (s) at ($(s) + (0,0.5)$);
    \node[minimum size=4.5mm, minimum width = 15mm, draw, rectangle] at (s) {5};
    \coordinate (s) at ($(s) + (1.5,-0.5)$);
    \node[minimum size=6mm, minimum width = 15mm, draw, rectangle] at (s) {V[5,6]};
    \coordinate (s) at ($(s) + (0,0.5)$);
    \node[minimum size=4.5mm, minimum width = 15mm, draw, rectangle] at (s) {6};
    \coordinate (s) at ($(s) + (1.5,-0.5)$);
    \node[minimum size=6mm, minimum width = 15mm, draw, rectangle] at (s) {V[7,8]};
    \coordinate (s) at ($(s) + (0,0.5)$);
    \node[minimum size=4.5mm, minimum width = 15mm, draw, rectangle] at (s) {7};
    \coordinate (s) at ($(s) + (1.5,-0.5)$);
    \node[minimum size=6mm, minimum width = 15mm, draw, rectangle] at (s) {V[1,1]};
    \coordinate (s) at ($(s) + (0,0.5)$);
    \node[minimum size=4.5mm, minimum width = 15mm, draw, rectangle] at (s) {8};
    \coordinate (s) at ($(s) + (1.5,-0.5)$);
    \node[minimum size=6mm, minimum width = 15mm, draw, rectangle] at (s) {\ldots};
    \coordinate (s) at ($(s) + (0,0.5)$);
    \node[minimum size=4.61mm, minimum width = 15mm, draw, rectangle] at (s) {\ldots};
    \coordinate (s) at ($(s) + (1.5,-0.5)$);
    \node[minimum size=6mm, minimum width = 15mm, draw, rectangle] at (s) {V[8,8]};
    \coordinate (s) at ($(s) + (0,0.5)$);
    \node[minimum size=4.5mm, minimum width = 15mm, draw, rectangle] at (s) {15};
    \coordinate (s) at ($(s) + (1.5,-0.5)$);
\end{tikzpicture}
\end{figure}

\end{subsection}

\end{section}

\end{document}